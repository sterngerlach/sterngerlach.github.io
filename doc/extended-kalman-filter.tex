
% extended-kalman-filter.tex

\documentclass[dvipdfmx,a4paper]{jsarticle}

\usepackage{docmute}

% settings.tex

\usepackage{atbegshi}
\AtBeginDvi{\special{pdf:tounicode 90ms-RKSJ-UCS2}}

\usepackage{amsmath}
\usepackage{amssymb}
\usepackage{amsfonts}
\usepackage{amsthm}
\usepackage{bm}
\usepackage{ascmac}
\usepackage{comment}
\usepackage{fancybox}
\usepackage{framed}
\usepackage{color}
\usepackage[dvipdfmx]{graphicx}
\usepackage{multicol}
\usepackage{multirow}
\usepackage{pdflscape}
\usepackage{verbatim}

\usepackage{url}
\urlstyle{same}

\DeclareMathOperator*{\argmax}{arg\,max}
\DeclareMathOperator*{\argmin}{arg\,min}
\DeclareMathOperator{\Tr}{Tr}
\DeclareMathOperator{\KL}{KL}
\DeclareMathOperator{\diag}{diag}
\DeclareMathOperator{\bel}{bel}
\DeclareMathOperator{\belp}{\overline{bel}}

\usepackage[T1]{fontenc}
\usepackage[utf8]{inputenc}

\usepackage{algorithm}
\usepackage{algpseudocode}
\usepackage{algorithmicx}

\renewcommand{\algorithmicrequire}{\textbf{Input:}}
\renewcommand{\algorithmicensure}{\textbf{Output:}}

\makeatletter
\renewcommand{\ALG@name}{アルゴリズム}
\makeatother

\algnewcommand{\Initialize}[1]{
	\State \textbf{Initialize:}
 	\State \hspace*{\algorithmicindent}\parbox[t]{0.8\linewidth}{\raggedright #1}}


\usepackage{geometry}
\geometry{left=19.05mm,right=19.05mm,top=19.05mm,bottom=19.05mm}

\title{拡張カルマンフィルタ}
\author{にゃーん}
\date{\today}

\begin{document}

\maketitle

この資料は、文献~\cite{Thrun07}の3.3節を基に作成されています。

\section{拡張カルマンフィルタ}
\subsection{拡張カルマンフィルタとは}
カルマンフィルタを適用するためには、状態遷移と計測のいずれも、ガウス雑音を含んだ\textbf{線形関数}として記述されなければならなかった。即ち、現在の状態$x_t$は直前の状態$x_{t - 1}$と制御動作$u_t$の線形関数、計測$z_t$は現在の状態$x_t$の線形関数である必要があった。この仮定を置くことによって、事後信念$\bel(x_t)$は全ての時刻においてガウス分布となり、事後信念の計算はガウス分布の2つのパラメータ(平均$\mu_t$と共分散行列$\Sigma_t$)の計算に置き換えられた。カルマンフィルタでは計算が簡略化される代わりに、状態遷移と観測の線形性が要求されるが、現実の世界においてこのような線形性が成り立つことは極めて稀である。例えばロボットの状態遷移の一つに回転運動が挙げられるが、回転運動の記述には非線形な三角関数が必要である。従って、線形性を前提とするカルマンフィルタは、ごく簡単な問題にしか適用できない不便なものとなる。\newline

\textbf{拡張カルマンフィルタ}(EKF: Extended Kalman Filter)では、状態遷移と計測における\textbf{線形性を仮定しない}。従って、現在の状態$x_t$と観測$z_t$は、以下のように\textbf{非線形関数}$g$と$h$を用いて記述されるとする。
\begin{eqnarray}
	x_t &=& g(x_{t - 1}, u_t) + \varepsilon_t \\
	z_t &=& h(x_t) + \delta_t
\end{eqnarray}
現在の状態$x_t$は、直前の状態$x_{t - 1}$と制御動作$u_t$についての非線形関数$g(x_{t - 1}, u_t)$に、雑音$\varepsilon_t$が加わったものとして記述される。$\varepsilon_t$は、平均が零ベクトルで共分散行列が$R_t$のガウス分布$\mathcal{N}(\varepsilon_t | 0, R_t)$に従う。計測$z_t$も、雑音$\delta_t$を含んだ、現在の状態$x_t$に関する非線形関数$h(x_t)$により記述される。$\delta_t$は、平均が零ベクトルで共分散行列が$Q_t$のガウス分布$\mathcal{N}(\delta_t | 0, Q_t)$に従う。雑音$\varepsilon_t$と$\delta_t$の定義は、カルマンフィルタのときと同様である。\newline

ガウス分布は、確率変数を線形変換してもガウス分布になるという性質をもつ。カルマンフィルタでは、状態遷移や計測が、雑音$\varepsilon_t, \delta_t$を含んだ線形関数として表現された。故に、雑音$\varepsilon_t, \delta_t$がガウス分布に従うとき、現在の状態$x_t$や計測$z_t$もやはりガウス分布に従っていた。しかし拡張カルマンフィルタでは、状態遷移と計測が非線形でもよいので、雑音$\varepsilon_t, \delta_t$が正規分布に従っていても、現在の状態$x_t$や計測$z_t$が正規分布に従うとは限らない。このとき信念分布$\bel(x_t)$はガウス分布にならず、信念分布を厳密に求めることは、もはや不可能になってしまう。それでは困るので、信念分布$\bel(x_t)$は平均$\mu_t$、共分散行列$\Sigma_t$のガウス分布$\mathcal{N}(x_t | \mu_t, \Sigma_t)$により\textbf{近似}される。拡張カルマンフィルタにおいても、信念分布の表現は、元のカルマンフィルタと同様である。但しカルマンフィルタのように、信念分布は厳密にガウス分布になるのではない。しかし事後信念の推定は、ガウス分布の2つのパラメータの推定に置き換えられるので、カルマンフィルタと同様、効率良く事後信念を計算できる。

\subsection{線形化}
\subsubsection{状態遷移の線形化}
信念分布がガウス分布となるためには、現在の状態$x_t$と観測$z_t$は、線形関数(線形ガウスモデル)として記述されなければならない。しかし拡張カルマンフィルタでは、$x_t$と$z_t$のいずれも\textbf{非線形関数}として表現されるので、何らかの方法で線形化する必要がある。拡張カルマンフィルタでは、1次の\textbf{テイラー展開}を用いて、非線形関数$g(x_{t - 1}, u_t)$と$h(x_t)$を線形化する。関数$g(x_{t - 1}, u_t)$を、状態$x$のまわりでテイラー展開すると次のようになる。
\begin{equation}
	g(x_{t - 1}, u_t) = g(x, u_t) + g'(x, u_t) \left( x_{t - 1} - x \right) + \frac{1}{2} \left( x_{t - 1} - x \right)^T g''(x, u_t) \left( x_{t - 1} - x \right) + \cdots
\end{equation}
ここでは$x_{t - 1}$と$u_t$に関する線形関数となればよいので、テイラー展開は1次の項までで打ち切る。
\begin{equation}
	g(x_{t - 1}, u_t) \simeq g(x, u_t) + g'(x, u_t) \left( x_{t - 1} - x \right)
\end{equation}
上式において$g'(x, u_t)$は次のように定義される。即ち関数$g(x_{t - 1}, u_t)$の$x_{t - 1}$に関する偏微分に、$x_{t - 1} = x$を代入したものである。状態ベクトル$x_t$の次元を$n$とすると、以下の$g'(x, u_t)$は$n \times n$の正方行列となり、この行列は\textbf{ヤコビ行列}とよばれる。
\begin{equation}
	g'(x, u_t) = \nabla g(x, u_t) = \left. \frac{\partial g(x_{t - 1}, u_t)}{\partial x_{t - 1}} \right|_{x_{t - 1} = x}
\end{equation}
上式の$x$には、以前の時刻$t - 1$において最も可能性の高い(最尤な)状態を選べばよい。時刻$t - 1$におけるロボットの状態$x_{t - 1}$の事後信念は、ガウス分布$\mathcal{N}(x_{t - 1} | \mu_{t - 1}, \Sigma_{t - 1})$により近似された。このガウス分布が最大となる平均$\mu_{t - 1}$が、直前の時刻における最尤な状態といえる。従って$x = \mu_{t - 1}$を関数$g$のテイラー展開の式に代入すれば、以下を得られる。即ち以下の式は、関数$g$を最尤な状態$\mu_{t - 1}$のまわりでテイラー展開し、1次の項まで求めたものである。最尤な状態のまわりでテイラー展開を行えば、近似された線形関数と、元の非線形関数との誤差を最小限に抑えられる。ヤコビ行列$g'(\mu_{t - 1}, u_t)$は定数であるから、これを$G_t$とおく。
\begin{eqnarray}
	g(x_{t - 1}, u_t) &\simeq& g(\mu_{t - 1}, u_t) + g'(\mu_{t - 1}, u_t) \left( x_{t - 1} - \mu_{t - 1} \right) \\
	&=& g(\mu_{t - 1}, u_t) + G_t \left( x_{t - 1} - \mu_{t - 1} \right)
\end{eqnarray}
この準備によって、現在の状態$x_t$は、直前の状態$x_{t - 1}$と制御$u_t$の線形関数として次のように近似される。即ちロボットの状態遷移は線形ガウスモデルで表現される。
\begin{eqnarray}
	x_t &=& g(x_{t - 1}, u_t) + \varepsilon_t \nonumber \\
	&\simeq& g(\mu_{t - 1}, u_t) + G_t \left( x_{t - 1} - \mu_{t - 1} \right) + \varepsilon_t
\end{eqnarray}
ガウス雑音$\varepsilon_t$は、平均が零ベクトルで分散が$R_t$のガウス分布$p(\varepsilon_t) = \mathcal{N}(\varepsilon_t | 0, R_t)$であった。雑音$\varepsilon_t$に関する確率分布$p(\varepsilon_t)$を変形して、状態遷移確率$p(x_t | x_{t - 1}, u_t)$を以下のように得る。
\begin{eqnarray}
	p(\varepsilon_t) &=& \mathcal{N}(\varepsilon_t | 0, R_t) \nonumber \\
	&=& \left| 2 \pi R_t \right|^{-\frac{1}{2}} \exp \left\{ -\frac{1}{2} \varepsilon_t^T R_t^{-1} \varepsilon_t \right\} \nonumber \\
	&=& \left| 2 \pi R_t \right|^{-\frac{1}{2}} \exp \left\{ -\frac{1}{2} \left( x_t - g(\mu_{t - 1}, u_t) - G_t \left( x_{t - 1} - \mu_{t - 1} \right) \right)^T R_t^{-1} \right. \nonumber \\
	&& \qquad \left( x_t - g(\mu_{t - 1}, u_t) - G_t \left( x_{t - 1} - \mu_{t - 1} \right) \right) \bigg\}
\end{eqnarray}
最後の式変形では$\varepsilon_t = x_t - g(\mu_{t - 1}, u_t) - G_t \left( x_{t - 1} - \mu_{t - 1} \right)$を代入している。これより$x_t$に関する分布の式が得られたので、状態遷移確率$p(x_t | x_{t - 1}, u_t)$は次のようになる。これは、平均$g(\mu_{t - 1}, u_t) + G_t \left( x_{t - 1} - \mu_{t - 1} \right)$で共分散行列$R_t$の多変量ガウス分布である。
\begin{eqnarray}
	p(x_t | x_{t - 1}, u_t) &=& \mathcal{N}(x_t | g(\mu_{t - 1}, u_t) + G_t \left( x_{t - 1} - \mu_{t - 1} \right), R_t) \\
	&=& \left| 2 \pi R_t \right|^{-\frac{1}{2}} \exp \left\{ -\frac{1}{2} \left( x_t - g(\mu_{t - 1}, u_t) - G_t \left( x_{t - 1} - \mu_{t - 1} \right) \right)^T R_t^{-1} \right. \nonumber \\
	&& \qquad \left( x_t - g(\mu_{t - 1}, u_t) - G_t \left( x_{t - 1} - \mu_{t - 1} \right) \right) \bigg\}
\end{eqnarray}

\subsubsection{計測の線形化}
非線形関数$h(x_t)$の線形化も、関数$g(x_{t - 1}, u_t)$と同様に行える。関数$h(x_t)$を、状態$x$のまわりで1次の項までテイラー展開する。
\begin{equation}
	h(x_t) \simeq h(x) + h'(x) \left( x_t - x \right)
\end{equation}
上式において$h'(x)$は、関数$h(x_t)$の$x_t$に関する偏微分に、$x_t = x$を代入したものである。計測ベクトル$z_t$の次元を$k$とすると、以下のヤコビ行列$h'(x)$は$k \times k$の正方行列となる。
\begin{equation}
	h'(x) = \nabla h(x) = \left. \frac{\partial h(x_t)}{\partial x_t} \right|_{x_t = x}
\end{equation}
上式の$x$としては、$z_t$の計測が起こる時点での尤もらしい状態を選べばよい。ここで扱っているロボットの確率モデルは、制御動作$u_t$の後に観測$z_t$が起こるものであった。即ち、ロボットに制御$u_t$が加わることで、直前の状態$x_{t - 1}$から新たな状態$x_t$への遷移が起こり、この状態遷移の後に最新の計測$z_t$が得られる。現在の状態$x_t$に関する信念は、次のように2段階に分けて更新された。制御$u_t$によって、直前の事後信念$\bel(x_{t - 1}) = \mathcal{N}(x_{t - 1} | \mu_{t - 1}, \Sigma_{t - 1})$から、現在の時刻における事前信念$\belp(x_t) = \mathcal{N}(x_t | \overline{\mu}_t, \overline{\Sigma}_t)$が計算され、現在の状態$x_t$に関する予測がなされる。続いて観測$z_t$により、$x_t$に関する事前信念$\belp(x_t)$から、$x_t$に関する事後信念$\bel(x_t) = \mathcal{N}(x_t | \mu_t, \Sigma_t)$が計算され、$x_t$に対する予測が修正される。従って、$z_t$が計測される時点での、ロボットが考えている最尤な状態は$\overline{\mu}_t$であるから、$\overline{\mu}_t$をテイラー展開の式に代入すればよい。
\begin{eqnarray}
	h(x) &\simeq& h(\overline{\mu}_t) + h'(\overline{\mu}_t) \left( x_t - \overline{\mu}_t \right) \\
	&=& h(\overline{\mu}_t) + H_t \left( x_t - \overline{\mu}_t \right)
\end{eqnarray}
上式は、関数$h$を現時点で最尤な状態$\overline{\mu}_t$のまわりで、1次の項までテイラー展開したものである。ヤコビ行列$h'(\overline{\mu}_t)$は定数であるから、これを$H_t$とおいた。この線形近似によって、現在の計測$z_t$は、現在の状態$x_t$の線形関数として近似的に表現される。
\begin{eqnarray}
	z_t &=& h(x_t) + \delta_t \nonumber \\
	&\simeq& h(\overline{\mu}_t) + H_t \left( x_t - \overline{\mu}_t \right) + \delta_t
\end{eqnarray}
ガウス雑音$\delta_t$は、平均が零ベクトルで共分散行列が$Q_t$のガウス分布$p(\delta_t) = \mathcal{N}(\delta_t | 0, Q_t)$に従う。雑音$\delta_t$に関する確率分布$p(\delta_t)$の式を変形すれば、計測確率$p(z_t | x_t)$は次のようになる。最後の式変形では$\delta_t = z_t - h(\overline{\mu}_t) - H_t \left( x_t - \overline{\mu}_t \right)$を代入している。
\begin{eqnarray}
	p(\delta_t) &=& \mathcal{N}(\delta_t | 0, Q_t) \nonumber \\
	&=& \left| 2 \pi Q_t \right|^{-\frac{1}{2}} \exp \left\{ -\frac{1}{2} \delta_t^T Q_t^{-1} \delta_t \right\} \\
	&=& \left| 2 \pi Q_t \right|^{-\frac{1}{2}} \exp \left\{ -\frac{1}{2} \left( z_t - h(\overline{\mu}_t) - H_t \left( x_t - \overline{\mu}_t \right) \right)^T Q_t^{-1} \left( z_t - h(\overline{\mu}_t) - H_t \left( x_t - \overline{\mu}_t \right) \right) \right\}
\end{eqnarray}
これより$z_t$に関する分布の式が得られた。計測確率$p(z_t | x_t)$は、平均$h(\overline{\mu}_t) + H_t \left( x_t - \overline{\mu}_t \right)$で共分散行列$Q_t$のガウス分布となる。
\begin{eqnarray}
	p(z_t | x_t) &=& \mathcal{N}(z_t | h(\overline{\mu}_t) + H_t \left( x_t - \overline{\mu}_t \right), Q_t) \\
	&=& \left| 2 \pi Q_t \right|^{-\frac{1}{2}} \exp \left\{ -\frac{1}{2} \left( z_t - h(\overline{\mu}_t) - H_t \left( x_t - \overline{\mu}_t \right) \right)^T Q_t^{-1} \left( z_t - h(\overline{\mu}_t) - H_t \left( x_t - \overline{\mu}_t \right) \right) \right\}
\end{eqnarray}
以上より、状態遷移確率$p(x_t | x_{t - 1}, u_t)$と計測確率$p(z_t | x_t)$を、カルマンフィルタと同じようにガウス分布で近似的に表現できた。従って、任意の時刻$t$において信念分布$\belp(x_t), \ \bel(x_t)$がガウス分布となり、元のカルマンフィルタと同様のアルゴリズムで、拡張カルマンフィルタの計算を効率良く進められるようになる。

\subsection{アルゴリズム}
拡張カルマンフィルタは以下のアルゴリズム\ref{alg:extended-kalman-filter}のように記述される。元のカルマンフィルタと同様に、時刻$t$における制御$u_t$、計測$z_t$、時刻$t - 1$における事後信念$\bel(x_{t - 1})$のパラメータ$\mu_{t - 1}, \Sigma_{t - 1}$を入力として取り、時刻$t$における事後信念$\bel(x_t)$を構成するパラメータ$\mu_t, \Sigma_t$を出力する。即ち時刻$t - 1$における事後信念$\bel(x_{t - 1})$から、現在の時刻$t$における事後信念$\bel(x_t)$を求める。

\begin{algorithm}[H]
	\caption{拡張カルマンフィルタ}
	\label{alg:extended-kalman-filter}
	\begin{algorithmic}[1]
		\Require
			\Statex 時刻$t - 1$における事後信念$\bel(x_{t - 1})$の平均$\mu_{t - 1}$と共分散行列$\Sigma_{t - 1}$
			\Statex 時刻$t$における制御$u_t$
			\Statex 時刻$t$における計測$z_t$
		\Ensure
			\Statex 時刻$t$における事後信念$\bel(x_t)$の平均$\mu_t$と共分散行列$\Sigma_t$ \newline

		\State 事前信念$\belp(x_t)$の平均$\overline{\mu}_t$を計算: $\overline{\mu}_t = g(\mu_{t - 1}, u_t)$ \label{alg:extended-kalman-filter-prediction-mu}
		\State 事前信念$\belp(x_t)$の共分散行列$\overline{\Sigma}_t$を計算: $\overline{\Sigma}_t = G_t \Sigma_{t - 1} G_t^T + R_t$ \label{alg:extended-kalman-filter-prediction-sigma}
		\State カルマンゲイン$K_t$を計算: $K_t = \overline{\Sigma}_t H_t^T \left( H_t \overline{\Sigma}_t H_t^T + Q_t \right)^{-1}$ \label{alg:extended-kalman-filter-gain}
		\State 事後信念$\bel(x_t)$の平均$\mu_t$を計算: $\mu_t = \overline{\mu}_t + K_t \left( z_t - h(\overline{\mu}_t) \right)$ \label{alg:extended-kalman-filter-correction-mu}
		\State 事後信念$\bel(x_t)$の共分散行列$\Sigma_t$を計算: $\Sigma_t = \left( I - K_t H_t \right) \overline{\Sigma}_t$ \label{alg:kalman-filter-correction-sigma}
	\end{algorithmic}
\end{algorithm}

元のカルマンフィルタのアルゴリズムと比較すると、殆ど大差ないことが分かる。\ref{alg:extended-kalman-filter-prediction-mu}行目における現在の状態の予測($\overline{\mu}_t$の計算)は、$\overline{\mu}_t = A_t \mu_{t - 1} + B_t u_t$のような線形関数ではなく、$\overline{\mu}_t = g(u_t, \mu_{t - 1})$という非線形関数に置き換わっている。また\ref{alg:extended-kalman-filter-correction-mu}行目における、(現在の状態予測$\overline{\mu}_t$に基づいた)計測$z_t$の予測も、$C_t \overline{\mu}_t$のような線形関数ではなく、$h(\overline{\mu}_t)$という非線形関数になっている。即ち、現在の状態$x_t$と計測$z_t$は、非線形関数$g, h$から予測される。\newline

またカルマンフィルタにおける行列$A_t$と$B_t$は、現在の状態予測$x_t$と、前時刻における状態$x_{t - 1}$および制御$u_t$とを結びつける係数である。拡張カルマンフィルタにおいては、ヤコビ行列$G_t$が、$A_t, B_t$の代わりに、これらの変数を関連付ける役割をもつ。カルマンフィルタにおける行列$C_t$は、現在の状態予測$x_t$と計測$z_t$とを結びつける変数であり、拡張カルマンフィルタにおいては、ヤコビ行列$H_t$が$C_t$と同様の役割を果たす。拡張カルマンフィルタのアルゴリズムの導出は、カルマンフィルタと殆ど同様に行えるため省略する。

\bibliographystyle{plain}
\bibliography{extended-kalman-filter}

\end{document}
